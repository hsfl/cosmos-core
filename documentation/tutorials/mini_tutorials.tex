\documentclass[10pt,letterpaper]{article}

% ----------
% include the standard packages for the COSMOS documentation
\usepackage[utf8]{inputenc}
\usepackage{amsmath}
\usepackage{amsfonts}
\usepackage{amssymb}

%\usepackage[left=.5in, right=.25in, top=0.5.00in, bottom=0.50in]{geometry}
%\usepackage[left=1.00cm, right=1.00cm, top=2cm,bottom=2.00cm]{geometry}
\usepackage[pdftex]{graphicx} 

\usepackage{hyperref}
\usepackage{listings} % for inline code formating
\usepackage{xcolor}
\usepackage{color}

\definecolor{Brown}{cmyk}{0,0.81,1,0.60}
\definecolor{OliveGreen}{cmyk}{0.64,0,0.95,0.40}
\definecolor{CadetBlue}{cmyk}{0.62,0.57,0.23,0}


% works 
%\lstset { %
%    language=C++,
%    backgroundcolor=\color{black!5}, % set backgroundcolor
%    basicstyle=\footnotesize,% basic font setting
%}

\lstset
{
	language=C++,
	backgroundcolor=\color{black!5}, % set backgroundcolor
	frame=ltrb,
	framesep=5pt,
	basicstyle=\normalsize,
	%basicstyle=\footnotesize,% basic font setting
	keywordstyle=\ttfamily\color{OliveGreen},
	identifierstyle=\ttfamily\color{CadetBlue}\bfseries,
	commentstyle=\color{Brown},
	stringstyle=\ttfamily,
	showstringspaces=ture
}

\lstdefinestyle{BashInputStyle}{
  language=bash,
  basicstyle=\small\sffamily,
  %numbers=left,
  %numberstyle=\tiny,
  %numbersep=3pt,
  frame=tb,
  columns=fullflexible,
  backgroundcolor=\color{yellow!20},
  linewidth=1\linewidth,
  xleftmargin=0\linewidth
}


%opening}

% ----------
% include any specific packages for this document
\usepackage[left=2.00cm, right=2.00cm, top=2cm,bottom=2.00cm]{geometry}

\title{Mini Tutorials on COSMOS-core}
%\author{Eric J. Pilger, Miguel A. Nunes}

\begin{document}

\maketitle
\tableofcontents

%\chapter{Mini Tutorials}


\newpage
\section{Add a new generic device}

As an example we are going to add a new generic device to measure the temperature named ``temperatureStation". Go to jsondef.h approx in line 960 and create the structure that  contains the information you want to use.

\begin{lstlisting}
struct temperatureStationStruc
{
	//! Generic info must be here for every device
	genstruc gen;
	//! the following is any data specific to this device
	float temperature; // your temperature data will be stored here
} ;

\end{lstlisting}

add your temperatureStationStruc structure to the devicestruc union (apporox in line 1400)

\begin{lstlisting}
typedef struct
{
	union
	{
		allstruc all;
		...
		thststruc thst;
		tsenstruc tsen;
		temperatureStationStruc temperatureStation; // << --- add here
	};
} devicestruc;

\end{lstlisting}


add your temperatureStationStruc structure to the devspecstruc structure (apporox in line 1500)


\begin{lstlisting}
typedef struct
{
	uint16_t ant_cnt;
	...
	uint16_t thst_cnt;
	uint16_t tsen_cnt; 
	uint16_t temperatureStation_cnt; // << --- add here
	vector<allstruc *>all;
	...
	vector<thststruc *>thst;
	vector<tsenstruc *>tsen;
	vector<temperatureStationStruc *>temperatureStation;  // << --- add here
} devspecstruc;

\end{lstlisting}


now go to jsonlib.cpp , add your temperatureStation to the end of device\_type\_string


\begin{lstlisting}
vector <string> device_type_string
{
	"pload",
	...
	"cam",
	"temperatureStation" // <-- add here
};


\end{lstlisting}

in jsondef.h you also must add the device type to the end of device\_type\ enum (approx in line 400)


\begin{lstlisting}
enum
	{
	//! Payload
	DEVICE_TYPE_PLOAD=0,
	...
	//! Camera
	DEVICE_TYPE_CAM=26,
	//! your tempStation here
	DEVICE_TYPE_TEMPSTATION = 27, // <- add here 
	//! List count
	DEVICE_TYPE_COUNT,
	//! Not a Component
	DEVICE_TYPE_NONE=65535
	};

\end{lstlisting}


now we are going to modify some functions in the code. The first one is json\_detroy in jsonlib.cpp


\begin{lstlisting}
void json_destroy(cosmosstruc *cdata)
{
	for (uint16_t i=0; i<2; ++i)
	{
		cdata[i].devspec.ant.resize(0);
		...
		cdata[i].devspec.tsen.resize(0);
		cdata[i].devspec.temperatureStation.resize(0); // <- add here
		cdata[i].device.resize(0);

	}

	delete [] cdata;
	cdata = NULL;
}

\end{lstlisting}

(side note: for a really complex type further definitions must be added to the namespace, but most common types are already supported, so this is an advanced feature)

go to json\_devices\_specific and inside the for loop that goes over each type add some of the following

\begin{lstlisting}
const char *json_devices_specific(string &jstring, cosmosstruc *cdata)
{
...

	for (uint16_t j=0; j<*cnt; ++j)
	{
	...
  
		// Dump Temperature Station
	 	if (!strcmp(device_type_string[i].c_str(),"tempStation"))
	 	{
		 	json_out_1d(jstring,(char *)"device_tempStation_temperature",j,cdata);
	 		json_out_character(jstring, '\n');
	 	}

	} 
}

\end{lstlisting}

go to json\_clone

\begin{lstlisting}

int32_t json_clone(cosmosstruc *cdata)
{
...
	case DEVICE_TYPE_TEMPSTATION:
		cdata[1].devspec.tempStation[cdata[1].device[i].all.gen.didx] =
			&cdata[1].device[i].tempStation;
		break;
...
}

\end{lstlisting}

add name for the device count in json\_addbaseentry

\begin{lstlisting}

uint16_t json_addbaseentry(cosmosstruc *cdata)
{

	...
	json_addentry("device_tempStation_cnt", 
		UINT16_MAX, 
		UINT16_MAX,
		offsetof(devspecstruc,tempStation_cnt), 
		COSMOS_SIZEOF(uint16_t), 
		(uint16_t)JSON_TYPE_UINT16,
		JSON_GROUP_DEVSPEC,
		cdata);

}

\end{lstlisting}


to json\_adddeviceentry add 

\begin{lstlisting}

uint16_t json_adddeviceentry(uint16_t i, cosmosstruc *cdata)
{

	...
	case DEVICE_TYPE_TEMPSTATION:
	
		json_addentry("device_tempStation_utc",
			didx, 
			UINT16_MAX, 
			(ptrdiff_t)offsetof(genstruc,utc)+i*sizeof(devicestruc),
			COSMOS_SIZEOF(double), 
			(uint16_t)JSON_TYPE_DOUBLE,
			JSON_GROUP_DEVICE,
			cdata);

		json_addentry("device_tempStation_temperature",
			didx, 
			UINT16_MAX,
			(ptrdiff_t)offsetof(temperatureStationStruc,temperature) + 
				i*sizeof(devicestruc),
			COSMOS_SIZEOF(double), 
			(uint16_t)JSON_TYPE_DOUBLE,
			JSON_GROUP_DEVICE,
			cdata);
		
		cdata[0].devspec.tempStation.push_back(
			(temperatureStationStruct *)&cdata[0].device[i].tempStation);
		cdata[0].devspec.tempStation_cnt = 
			(uint16_t)cdata[0].devspec.tempStation.size();
	break;

}

\end{lstlisting}



% --------------------------------------------------------------------

%\newpage
%\section{Add a new common device}
%
%overview
%\begin{itemize}
%\item add to json
%\item add to simulator
%\item add to node.ini creator
%\end{itemize}
%
%All things are parts
%Parts - everything in nodes are parts, physical size and location
%Components - electrical parts - nominal voltage,current, actual voltage and current, some components are 
%Devices - specialized component
%
%
%\subsection{edit jsondef.h}
%
%
%\begin{lstlisting}
%//! Maximum number of Star Trackers
%#define MAXSTT 2
%
%int16_t stt_cnt; // star tracker count, how many STTs there are?
%
%// static 
%sttstruc_s stt[MAXSTT];
%
%// dynamic
%sttstruc_d stt[MAXSTT];
%\end{lstlisting}
%
%\begin{lstlisting}
%//! Star Tracker (STT) Static Structure
%typedef struct
%    {
%    int16_t cidx;    // component index
%    quaternion algn; // alignment quaternion
%    } sttstruc_s;
%
%//! Star Tracker (STT) Dynamic Sructure
%typedef struct
%    {
%    //double utc; // probably not needed
%    qatt att; // includes 0 and 1st order derivative
%    } sttstruc_d;
%\end{lstlisting}
%
%
%
%\subsection{edit jsonlib.cpp}
%\begin{lstlisting}
%
%{"stt_cidx",JSON_TYPE_INT16},
%{"stt_algn",JSON_TYPE_QUATERNION},
%{"stt_att",JSON_TYPE_QATT},    
%{"stt_cnt",JSON_TYPE_INT16},// how many of these
%
%
%//Device: Star Tracker
%json_addentry("stt_cnt",-1,-1,offsetof(cosmosstruc,stat.node.stt_cnt));
%for (i=0; i<MAXIMU; i++)
%    {
%    json_addentry("stt_cidx",i,-1,
%    offsetof(cosmosstruc,stat.node.stt)+
%    (ptrdiff_t)offsetof(sttstruc_s,cidx)+i*sizeof(sttstruc_s));
%    
%    json_addentry("stt_algn",i,-1,
%    offsetof(cosmosstruc,stat.node.stt)+
%    (ptrdiff_t)offsetof(sttstruc_s,algn)+i*sizeof(sttstruc_s));
%    
%    json_addentry("stt_att",i,-1,
%    offsetof(cosmosstruc,dyn.node.stt)+
%    (ptrdiff_t)offsetof(sttstruc_d,att)+i*sizeof(sttstruc_d));
%    
%    }
%
%//i index for each device, ex: stt_cidx_001...
%//-1, no second dimension
%//offsetof, memory mapping
%
%
%add to json_satellite
%
%// Dump STT's
%json_out(jstring,(char *)"stt_cnt",sroot,droot);
%for (i=0; i<*(int16_t *)json_ptrto((char *)"stt_cnt",sroot,droot); i++)
%    {
%    json_out_1d(jstring,(char *)"stt_cidx",i,sroot,droot);
%    json_out_1d(jstring,(char *)"stt_algn",i,sroot,droot);
%    }
%
%\end{lstlisting}
%
%
%\subsection{edit satlib.cpp}
%\begin{lstlisting}
%in void load_databases()
%
%sttstruc_d *std;
%sttstruc_s *sts;
%
%
%op = fopen("stt.txt","r");
%if (op != NULL)
%{
%for (i=0; i<cosmos_data.stat.node.stt_cnt; i++)
%	{
%	sts = &cosmos_data.stat.node.stt[i];
%	std = &cosmos_data.dyn.node.stt[i];
%	fscanf(op,"%*d\t%hd\t%lg\t%lg\t%lg\t%lg\t%lg\t%lg\t%lg\t%lg\t%lg\t%lg\t%lg\t%lg\t%lg\t%lg\n",
%	&sts->cidx,
%	&sts->algn.d.x,
%	&sts->algn.d.y,
%	&sts->algn.d.z,
%	&sts->algn.w,
%	&std->att.s.d.x,
%	&std->att.s.d.y,
%	&std->att.s.d.z,
%	&std->att.s.w,
%	&std->att.v.col[0],
%	&std->att.v.col[1],
%	&std->att.v.col[2],
%	&std->att.a.col[0],
%	&std->att.a.col[1],
%	&std->att.a.col[2]);
%	}
%fclose(op);
%}
%\end{lstlisting}
%
%\subsection{create node.ini (Eric's Magic)}
%
%make stt.txt file with tab delimited fields



%\newpage
%\section{extending the COSMOS namespace}
%
%example on extending the COSMOS namespace by adding port/address name to the devices namespace.
%
%\subsection{edit nodedef.h}
%
%look for nodestruc\_s \\
%look for imustruc\_s, line 220 \\
%add connection information for device:
%\begin{lstlisting}
%char port[COSMOS_MAX_NAME];
%\end{lstlisting}
%note: "COSMOS\_MAX\_NAME" contains 40 spaces \\
%add this to other devices that may require the name information:
%\begin{itemize}
%\item imu
%\item stt
%\item rw
%\item tcu (mtr)
%\item gps
%\item payload
%\item cpu
%\end{itemize}
%
%now recompile and check everything works
%\subsection{edit jsonlib.c}
%add the following JSON strings to the COSMOS namespace table
%\begin{lstlisting}
%("cpu_port",JSON_\TYPE_\STRING)
%("gps_port",JSON_\TYPE_\STRING)
%("imu_port",JSON_\TYPE_\STRING)
%("stt_port",JSON_\TYPE_\STRING)
%("mtr_port",JSON_\TYPE_\STRING)
%("payload_port",JSON_\TYPE_\STRING)
%("rw_port",JSON_\TYPE_\STRING)
%\end{lstlisting}
%
%in jason\_setup(), line 2600, add entries by copying a static one:
%\begin{lstlisting}
%    json_addentry("cpu_port",i,-1,
%    offsetof(cosmosstruc,stat.node.stt)+
%    (ptrdiff_t)offsetof(cpustruc_s,algn)+i*sizeof(cpustruc_s));
%\end{lstlisting}
%
%do the same to the: imu, stt, gps, mtr, rw, payload.
%
%\subsection{edit node.ini}
%add {"imu\_port","/dev/ttyUSB0"} \\
%note: later this will be done automatically using the COSMOS editor that Kyle is working on.\\
%to finally use the port name, ex.: \ microstrain\_connect(cosmos\_data.stat.imu[0].port)\\
%
%----------------\\
%things to clarify\\
%COSMOS namespace vs C++ namespace\\
%
%
%notes:
%node.ini
%is a static description of the node
%
%
%all the node.ini information goes into the node\_s structure
%
%node\_s - description info about node
%node\_d - dynamic part of node, like telemetry


\newpage
\section{Software profiler}

\subsection{Linux}
To check how your software preforms in Linux you can use 'gprof'

1) Compile with correct switches 
-pg

CFLAGS = -pg 

go to examples/profiler
\$ make testprofiler

2) run to completion, exit normally 
this will create file gmon.out

3) gprof <program>
it reads gmon.out and prints a report

\subsection{Mac OS}
To profile on the the Mac use ``Instruments" budled with Xcode
or install
http://valgrind.org

Here is a list of profiling tools recomended by Qt: \\
\url{http://qt-project.org/wiki/Profiling-and-Memory-Checking-Tools}
%\href{http://qt-project.org/wiki/Profiling-and-Memory-Checking-Tools}{wikibooks home}



% Tristan Tutorial on Installing Latex and then Qt with MinGW
%\section{Installing Latex and Qt with MinGW}
%
%
%
%\subsection{Custom Qt , MinGW and CMake}
%Windows only:
%
%Download Qt4.8.5.zip, and the cmake, mingw and qt-5.2.0 executables from here: \url{http://cosmos-project.org/software/downloads/}
%
%- Run the Qt5 executable and an installation wizard will guide you through installation. Accept all defaults.
%
%- Run the cmake executable, accepting all defaults.
%
%- Run the mingw-builds executable, accepting all defaults.
%
%- Expand the Qt4.8.5 zip and place it in the Qt folder created on the C: drive by the Qt5 install.
%\\
%\\
%When you first run Qt Creator, you will need to go to the Build \& Run section of Tools:Options to set up your environment. You will need to point to CMake, Compilers, and Qt Versions. Browse, or add then browse, as appropriate. The paths should be something like:
%
%- CMake: C:\textbackslash Program Files (x86)\textbackslash CMake 2.8\textbackslash bin\textbackslash cmake.exe
%
%- Compilers: C:\textbackslash Program Files (x86)\textbackslash mingw-builds\textbackslash x32-4.8.1-posix-dwarf-rev5\textbackslash mingw32\textbackslash bin\textbackslash g++.exe
%
%- Qt Versions: C:\textbackslash Qt\textbackslash Qt4.8.5\textbackslash bin\textbackslash qmake.exe
%\\
%\\
%Once this has been established, go to the Kits tab and set up a custom kit by selecting the things you just added.






\end{document}
